\documentclass{article}
\usepackage[a4paper, margin=0.5in]{geometry}
\usepackage{tikz}
\usetikzlibrary{shapes.geometric, arrows}
\usepackage{mathtools}
\usepackage{tabto}
\usepackage{fontspec}
\date{}
\author{hacker}
\newfontfamily{\code}{Source Code Pro}
\begin{document}
	\title{Constants, Variables and Datatypes}
	\maketitle
	
	\newpage
	
	\section{Charecter Set}
	The Charecter Set in C is grouped into following categories : \\
	$ \bullet $ Letters (a to z and A to Z) \\
	$ \bullet $ Digits (0 to 9) \\
	$ \bullet $ Special Charecters (\#, ( ), \{ \}) \\
	$ \bullet $ White Spaces (Tab, Spacebar)
	
	\section{Trigraph}
	Each trigraph sequence consist of three charecters ('??' followed by the another chrecter) \\
	$ \bullet $ ??= $\rightarrow$ \# \\
	$ \bullet $ ??( $\rightarrow ~ $  [ \\
	$ \bullet $ ??) $\rightarrow ~ $  ] \\
	$ \bullet $ ??> $\rightarrow$ \} \\
	$ \bullet $ ??< $\rightarrow$ \{
	
	\section{C tokens}
	The smallest individual units are known as C tokens. \\
	C has six types of tokens :
	
	\tikzstyle{box} = [rectangle,text centered, draw=black]
	\tikzstyle{arrow} = [thick, ->, =stealth]
	
	\begin{tikzpicture}[node distance=2cm]
		\node[box] (1) {C tokens};
		\node[below of=1] (2) {};
		\node[box, left of=2, align=center] (3) {Constants \\ 3.14 \\ -5.6};
		\node[box, right of=2, align=center] (4) {Strings \\ zcat \\ FBI};
		
		\node[box, right of=4, xshift=1cm, align=center] (5) {Operators \\ / * + -};
		\node [box, right of=5, xshift=1cm,align=center, align=center] (6) {Special Symbols \\ \{\} \# []};
		
		\node[box, left of=3, xshift=-1cm, align=center] (7) {Keywords \\ for \\ while \\ float};
		\node[box, left of=7, xshift=-1cm, align=center] (8) {Identifiers \\ main \\ amount};
		
		\draw [arrow] (1) -| (5);
		\draw [arrow] (1) -| (6);
		\draw [arrow] (1) -| (3);
		\draw [arrow] (1) -| (4);
		\draw [arrow] (1) -| (8);
		\draw [arrow] (1) -| (7);
	\end{tikzpicture}
	
	\subsection{Constants}
	Constants in C refer to fixed values that do not change during execution of program. 
	\\ \\ 
	C supports following types of constansts : \\
	
	\begin{tikzpicture}[node distance=1.5cm]
		\node[box] (1) {Constants};
		\node[below of= 1] (2) {};
		
		\node[box, xshift=-1cm ,left of=2, align=center] (3) {Numeric Constants \\ Integer constansts \\ Real constants};
		
		\node[box, xshift=1cm, right of=2, align=center] (4) {Characteristic Constants \\ Single Character constants \\ String constants};
		
		\draw [arrow] (1) -| (3);
		\draw [arrow] (1) -| (4);
	\end{tikzpicture}
	
	\subsubsection{Integer constants}
	An integer constant consist of a sequence of digits. 
	\\ \\
	There are 3 types of integers : \\
	$ \bullet $ decimal integer \\
	$ \bullet $ Octal integer \\
	$ \bullet $ Hexadecimal Integer
	\\ \\
	\textbf{3.1.1.1 Decimal integer} \\
	It consist of numbers from 0 to 9, proceed by an optional + or - sign.
	\\ \\
	\textbf{Example} \\
	123, -321, 0, 654321, +78 
	\\ \\
	\textbf{3.1.1.2 Octal Integer} \\
	An Octal Integer constant consist of any combination of digits fro, 0 to 7, with leading 0.
	\\ \\
	\textbf{Example} \\
	037, 0 0435, 0551
	\\ \\
	\textbf{3.1.1.3 Hexadecimal Integer} \\
	A sequence of digits from which will have 0x or 0X is considered as hexamdecimal integer they may also include alphabets A to F or a to f. The letters a to f represents number 10 to 15.
	\\ \\
	\textbf{Example} \\
	0x2, 0x9f, 0XBCD, 0XF08
	\\ \\
	\textbf{3.1.1.4 Unassigned Integer} \\
	It is possible to store large number constants on these machines by appending qualifiers such as U, L and UL to the constants
	\\ \\
	\textbf{Examples} \\
	56789U, 987612347UL, 9876543l
	
	\subsubsection{Real constansts}
	Real numbers are constants containing fractional parts like 17.548 (or floating point) 
	\\ \\
	\textbf{Example} \\
	0.0083, -0.75, 435.36, +247.0
	\\ \\
	Real number may also be expressed interms exponential notation.
	\\ \\
	\textbf{Example} \\
	215.65 can be written as 21565e2 in exponential notation \\
	where e is known as mantisa
	\\ \\
	\textbf{Genaral form} \\
	mantisa e exponent
	\\ \\
	where, mantisa is either a real number expressed in decimal notation or an integer. \\
	exponent is an integer with an optional + or - sign. The letter '\textit{e}' sepatrating the mantisa can be written in lower or upper case.
	\\ \\
	\textbf{Example} \\
	0.654e4, 12e-2, 1.5e+5, 3018E3, -1.2E-1
	
	\newpage
	
	\section{Strings}
	
	\subsubsection{Single Chracter Constanst}
	It containes a single character with in a pair of single quote marks and character constants have integer values knowm as ASCII values.
	\\ \\
	\textbf{Example} \\
	'5', 'x', '.', ''
	
	\subsubsection{String Constants}
	A string constants are groups of characters enclosed in double quotes. \\
	The characters may be letters, numbers, special characters and blank space
	\\ \\
	\textbf{Example}
	"Hello", "1987", "Well done", "? ...", "5+3", "x"
	\\ \\
	Difference between single character constant and string constant \\
	\begin{tabular}{|c|c|}
		\hline
		Single Character Constant & String Constants \\
		\hline
		$\bullet$ It consist of single character \hspace{1cm} & $ \bullet$ It consist of group of characters \\
		
		$\bullet$ It is enclosed in single quotes \hspace{0.85cm} & $\bullet$ It is enclosed in double quotes \hspace{0.145cm} \\

		$\bullet$ Example : 'O', 'x', 'F' \hspace{1.945cm} & \hspace{0.095cm} $\bullet$ Example : "kill", "check", "chsh" \\
		\hline
	\end{tabular}
	\\ \\
	\textbf{Backlash Character Constant} \\
	C support special backlash character constants that are used in output functions.
	\\ \\
	These character combination are known as \underline{escape sequences}
	\\ \\
	\begin{tabular}{|c|c|}
		\hline
		Constants & Escaple Sequence \\
		\hline
		\textbackslash a & audible bell \\
		\textbackslash b & backspace \\
		\textbackslash f & form feed \\
		\textbackslash n & newline \\
		\textbackslash r & carriage return \\
		\textbackslash t & horizontal tab \\
		\textbackslash v & verticle tab \\
		\textbackslash ' & single quote \\
		\textbackslash " & double quote \\
		\textbackslash ? & question mark \\
		\textbackslash\textbackslash & backslash \\
		\textbackslash 0 & Null \\
		\hline
	\end{tabular}

	\subsubsection{Keywords}
	All keywords has fixed meanings and these cannot be changed. Keywords serve as basic building blocks of for program statements. All keywords must be written in lowercase.
	\\ \\
	\textbf{Example} \\
	char, long, float, if, else
	
	\subsubsection{Identifiers}
	They refer to the names of the variables, functions and arrays. These are user defined data-names and consist of letters \& digits.
	\\ \\
	\textbf{Rules for identifiers :} \\
	$\bullet$ First charecter must be an alphabet or \_ (underscore) \\
	$\bullet$ It must consist of only letters, digits and underscore \\
	$\bullet$ Only first 31 charecter \\
	$\bullet$ Cannot use a keyword \\
	$\bullet$ Must not contain a whitespace \\
	$\bullet$ Successive underscores are not valid \\
	$\bullet$ Uppercase and lower case letters are significant
	\\ \\
	\textbf{Example} \\
	mobile\_123, \_test
	
	\section{Variables}
	A variable is a data name whose value goes on changing during the execution of a program. \\
	A varibale may take different values during execution.
	\\ \\
	\textbf{Example} \\
	avg, -avg, -test\_1
	\\ \\
	\textbf{Rules for naming the varible} \\
	$\bullet$ They must be begin with a letter. Some systems may permit underscore as first character \\
	$\bullet$ The first 31 charecters are valid, some compilers may consider first 8 charecters only \\
	$\bullet$ Uppercase and lowercase are significant \\
	$\bullet$ It should not be a keyword \\
	$\bullet$ Whitespace is not allowed
	
	\newpage
	
	\section{Data types}
	Data types may represent the type of data that the variable hold
	
	\subsection{ANSI C supports the 3 types of data types}
	$\bullet$ Primary data types \\
	$\bullet$ Derived data types \\
	$\bullet$ User defined data types
	
	\subsection{Primary or Fundamental data types}
	$\bullet$ integer \\
	$\bullet$ floating point \\
	$\bullet$ double precision \\
	$\bullet$ void
	
	\subsection{Range of data types}
	\begin{tabular}{|c|c|}
		\hline
		Data type & Range \\
		\hline
		int & -32,768 to 32,767 \\
		float & 3.4e-38 to 3.4e38 \\
		char & -128 to 127 \\
		double & 1.7e-308 to 1.7e308 \\
		\hline
	\end{tabular}

	\subsubsection{int data type}
	It has a range between -32,768 to 32,767 for 16 bit machines and for 32 bit machine the range is from -2147,483,684 to 2,147,483,647.
	\\ \\
	\textbf{C has 3 types of integer storage} \\
	$\bullet$ int \\
	$\bullet$ short int \\
	$\bullet$ long int
	\\ \\
	\textbf{Example} \\
	int exec; exec=287;
	
	\subsubsection{floating point data type}
	The keyword is float. The range is between 3.4e-38 to 3.4e38 for 16 bit machine.
	\\ \\
	\textbf{Example} \\
	int n; n=298.87, n=0x98.22;
	
	\subsubsection{Double precision floating point data type}
	The keyword is double. The range is between 1.7e-308 to 1.7e308 for 16 bit machine,. \\
	The double data type is same as that of float but represent with greater precision.
	
	\subsubsection{Void data type}
	The keyword is void. The Void data type has no values. It is used to specify a type of a function is said to be void, when it does not returnany value to the calling function.
	
	\subsubsection{Character data type}
	The keyword is char. The range is between -128 to 127. Charecters are usually stored in 8 bits i.e one byte of internal storage.
	
	\newpage
	
	\section{Declaration of Variables}
	\textbf{Declaration does 3 things} \\
	$\bullet$ It tells the compiler what variable name is \\
	$\bullet$ It specify what type of data variable it hold \\
	$\bullet$ It allocates requiered amount of memory
	
	\subsection{Primary type declaration}
	The syntax is "data\_type v1, v2, v3 $\dots$ vn" \\
	where, \\
	\tab \tab data\_type \hspace{1.525cm} $\longrightarrow$ type of data \\
	\tab \tab v1, v2, v3 $\cdots$ vn" \hspace{0.394cm} $\longrightarrow$  name of the variables
	\\ \\
	$\bullet$ variables are separated by commas. \\
	$\bullet$ A declaration statement should end with semicloln.
	\\ \\ 
	\textbf{Example} \\
	int count; \\
	char kill;
	float 77.77;
	
	\subsection{User-defined Type declaration}
	C supports a feature known as "type definition" that allows users to definean identifier that would define the variables.
	\\ \\
	\textbf{Genaral syntax} \\ 
	\tab \tab typedef type identifier;
	\\ \\
	Where, type refers to an existing data type refers to new name given to the data type.
	\\ \\
	\textbf{Example} \\
	typedef int units; \\
	\tab \tab units ClassA, ClassB;
	\\ \\
	typedef float marks; \\
	\tab \tab marks m1,m2;
	
	\subsubsection{Enumbered data type}
	The identifier is a user defined enumarated data types, which can be used to declare variables that can be enclosed within the braces i.e enumarated constants
	\\ \\
	\textbf{Genaral syntax} \\
	\tab \tab enum identifier \{value1, value2, value3, $\dots$ ,value\textit{n}\};
	\\ \\
	After this we can declare variables as \\
	\tab \tab enum identifier v1,v2,v3, \dots vn;
	\\ \\
	The enumbered variables have one of the values of value1, value2, value3, $\dots$ ,value\textit{n}
	
	\newpage
	
	\subsubsection{Assignment statements}
	Values can be assigned using,
	\tab \tab vauable\_name=constant;
	\\ \\
	Where, \\
	variable\_name $\rightarrow$ name of variable \\
	constant $\rightarrow$ fixed value 
	\\ \\
	\textbf{Example}
	max\_val=1; \\ min\_val=0;
	\\ \\
	It is possible to assign a value to a variable at the time of variable declaration.
	\\ \\
	\textbf{Genaral syntax} \\
	data\_type variable\_name=constant;
	\\ \\
	\textbf{Example} \\ 
	int final\_value=100; \\
	char yes='x'; \\
	double test=86.45;
	\\ \\
	\textbf{Initialization} \\
	The process of giving initial values to variables.
	\\ \\
	\textbf{Example} \\
	x=y=z; \\
	pass=done='t'; \\
	a=b=c=ALPHABET;
	
	\section{Reading data from keyboard}
	Using scanf() function we can read the data from the keyboard.
	\\ \\
	\textbf{Genaral syntax} \\
	scanf("control\_string", \&variable1, \&variable2, \dots , \&variablen); 
	\\ \\
	The control string containes the format of data being recived. \\
	The \& symbol represent specify the variable name addres. 
	\\ \\
	\textbf{Example} \\
	scanf("\%d", \&number); \\
	scanf("\%d\%d",\&var1, \&var2); \\
	scanf("\%d\%f",\&number, \&precise);
	
	\section{Defining symbolic constants}
	\textbf{Genaral syntax} \\
	\#define symbolic\_name value\_of\_constant
	\\ \\
	Where, \\
	\tab \tab \#define $\rightarrow$ keyword \\
	\tab \tab symbolic\_name $\rightarrow$ name of the variable \\
	\tab \tab value\_of\_constant $\rightarrow$ fixed value
	\\ \\
	symbolic names are sometimes also called as constant identifiers 
	\\ \\
	\textbf{Example} \\
	\#define health 100 \\
	\#define PI 3.14 \\
	\#define MAX 2000
	
	\subsection{Rules for naming symbolic constants}
	$\bullet$ Symbolic names have some form of variable names. \\
	$\bullet$ No blank space between the '\#' '\&' 'define' \\
	$\bullet$ '\#' must be first character in the line \\
	$\bullet$ A blank space is requiered between '\#define', symbolic\_name' \& 'constant' \\
	$\bullet$ '\#define' statement must not end with semicolon \\
	$\bullet$ symbolic\_names are not declared for data\_types \\
	$\bullet$ It's data type depends on types of constants. \\
	$\bullet$ '\#define' statements may be appear anywhere in the program, but before it is referenced in the program
	\\ \\
	\textbf{Program to describle welcome message} \\
	{
		\code
		\#include<stdio.h> \\
		main () \{
		\tabto{0.5cm} printf("\textbackslash n\textbackslash nWelcome to C!");\\
		\}
	} \\ \\
	\textbf{Program to find addition of two numbers} \\
	{ \code
		\#include<stdio.h> \\
		int n1,n2,sum; \\
		main () \{
		\tabto{0.5cm} printf("\textbackslash nEnter two numbers :\textbackslash n>")\\
		\}
	}

	\section{Operators and expressions}
	\subsection{Introduction}
	An operator is a symbol that  tells the computer to perform mathematical or logical manipulations. Operators are used in programs to manipulate data \& variables.
	\\ \\
	\textbf{Classification of C operators} \\
	$\bullet$ Arithmetic operators \\
	$\bullet$ Logical operators \\
	$\bullet$ Assignment operator \\
	$\bullet$ Increment and decrement operator \\
	$\bullet$ Conditional operator \\
	$\bullet$ Bitwise operator \\
	$\bullet$ Special operator
	
	\subsection{Arithmetic operators}
	\begin{tabular}{|c|c|}
		\hline
		operator & meaning \\
		\hline
		+ & Addition or unary plus \\
		- & Minus or unary minus \\
		* & Multiplication \\
		/ & Division \\
		\% & Modulo division \\
		\hline
	\end{tabular} \\ \\ \\
	Integer division truncates any fractional part \\ 
	The modulo division cannot be used for floating point
	\\ \\
	\textbf{Example} \\
	a-b;, a*b;, a\%b;
	
	\newpage
	
	\subsubsection{Integer arithmetic}
	Both the operands in a single arithmetic expression such as a+b are integers. \\
	The above operation is called as Integer arithmetic operation. \\	
	For modulo operations sign of the result is always the sign of the first operand i.e \\
	-14\%3=-2 \\
	-14\%-3=-2 \\
	14\%-3=2
	\\ \\
	\textbf{Example} \\
	int a,b; \\ a+b; a*b; a/b; \\ \\
	suppose if a=14 \& b=4 then, \\ a+b=18 \\a-b=10 \\ a*b=14(4)=56 \\ a/b=14/4=3
	\\ \\
	$[$ Note $]$ : 6/7=0 and -6/-7=0 but -6/7 may be 0 or -1 depending on the machine.
	
	\subsubsection{Real arithmetic}
	An arithmetic operation involving only real operatands is called Real Arithmetic. \\
	A real operands assume values either in decimal or exponential notation.
	\\ \\
	\textbf{Example} \\
	x=6.0/7.0=0.857143 \\ y=1.0/3.0=0.333333 \\ z=-2.0/3.0=0.666667
	\\ \\
	$[$ Note $]$ : The \% operand cannot be used with real operands.
	
	\subsubsection{Mixed mode arithmetic}
	One of the operand is real and other is integer the expression is called Mixed mode arithmetic.
	\\ \\
	\textbf{Example} \\
	15/10.0=1.5 \\ 15/10=1
	
	\subsubsection{Relational operators}
	An expression containing a relation operator is called Relational expression.
	\\ \\
	\begin{tabular}{|c|c|}
		\hline
		operator & meaning \\
		\hline
		< & less than \\
		> & greater than \\
		<= & less than or equal to \\
		>= & greater than or equal to \\
		!= & not equal to \\
		== & equal to \\
		\hline
	\end{tabular}
	\\ \\ \\
	Syntax for relational expression is \\
	\tab \tab ae\_1 relational\_operator ae\_2
	\\ \\
	where, \\
	\tab \tab ae\_1 and ae\_2 $\longrightarrow$ arithmetic expressions \\
	which may be constant variable or combination of them.
	\\ \\
	Suppose a=25, b=35; then, \\
	\begin{tabular}{cccc}
		a>b & F & a<b & T \\
		b>a & T & b<a & F \\
		a>=b & T & a<=b & F \\
		b>=a & T & b<=a & F \\
		a==b & F & a!=b & T
	\end{tabular}
	
	\subsubsection{Logical operators}
	An expression which contains logical operators are known as logical expression or a compound relational expression,
	\\ \\
	\textbf{Classification of Logical C operators} \\
	$\bullet$ Logical AND (\&\&) \\
	$\bullet$ Logical OR (||)\\
	$\bullet$ Logical NOT (!)
	\\ \\
	The logical operators \&\& \& and || are used when we want to test more than one conditions and make decision
	\\ \\
	\textbf{Example} \\
	a<b || a>b \\
	a==b \&\& a>=b
	\\ \\
	\textbf{Truth table} \\
	Some example of usage of large expression are \\
	if(age>55 \&\& salary<1000) \\
	if(number<0 || number>0)
	\\ \\
	\textbf{Logical AND}\\
	operend1 \&\& operend2 = operation
	\\ \\
	\begin{tabular}{|c|c|c|}
		\hline
		operend1 & operend2 & operation\\
		\hline
		True & True & True \\
		True & False & False \\
		False & True & False \\
		False & False & False \\
		\hline
	\end{tabular}
	\\ \\
	\textbf{Logical OR}\\
	operend1 || operend2 = operation
	\\ \\
	\begin{tabular}{|c|c|c|}
		\hline
		operend1 & operend2 & operation\\
		\hline
		True & True & True \\
		True & False & True \\
		False & True & True \\
		False & False & False \\
		\hline
	\end{tabular}
	\\ \\
	\textbf{Logical NOT} \\
	operand ! = operation
	\\ \\
	\begin{tabular}{|c|c|}
		\hline
		opernad & operation \\
		\hline
		True & False \\
		False & True \\
		\hline
	\end{tabular}

	\newpage

	\subsubsection{Assignment operators}
	It is used to assign the result of expression to a variable.
	\\ \\
	C has set of short hand assignment operator i.e \\
	\tab \tab v op=exp;
	\\ \\
	where, \\
	\tab \tab v $\longrightarrow$ variable \\
	\tab \tab op $\longrightarrow$ arithmetic operator \\
	\tab \tab exp $\longrightarrow$ expression
	\\ \\
	v=v op(exp);
	\\ \\
	\textbf{Example} \\
	a+=1; // same as a=a+1 \\
	\\ \\
	\textbf{Simple vs Shorthand assignment} \\ \\
	\begin{tabular}{|c|c|}
		\hline
		Simple assignment operator statement & Shorthand assignment statement \\
		\hline
		a=a+1; & a +=1; \\
		a=a-1; & a-=1; \\
		a=a*(n+1); & a*=n+1; \\
		a=a/(n+1); & a/=n+1; \\
		a=a\%b; & a\%=b; \\
		\hline		
	\end{tabular}
	\\ \\
	\textbf{Use of Shorthand assignment operator} \\
	$\bullet$ What apears on the left hand side need not to apear \\
	$\bullet$ The statement more concise ans easier to read \\
	$\bullet$ The statement is much easier
	\\ \\
	\textbf{Difference between = \& ==}
	\\ \\
	\begin{tabular}{|c|c|}
		\hline
		= & == \\
		\hline
		It is a Assignment operator $\quad \quad \quad \quad ~$ & It is a Relational operator $\quad \quad \quad \quad \quad \quad \quad \quad \quad ~ ~$\\
		Used to assign the values to a variable & Used to compare the variable (expression) values \\
		Example $\quad \quad \quad \quad \quad \quad \quad \quad \quad \quad \quad \quad ~~ $ & Example $$\quad \quad \quad \quad \quad \quad \quad \quad \quad \quad \quad \quad \quad \quad \quad \quad ~~~ $$ \\
		a=a++; , a=x-1; $ \quad \quad \quad \quad \quad \quad \quad \quad \quad $ & a==pow(x,y); , a==(x+y); $ \quad \quad \quad \quad \quad \quad \quad \quad \quad $ \\
		\hline
	\end{tabular}
	
	
\end{document} 