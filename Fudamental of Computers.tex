\documentclass{article}
\usepackage[a4paper, margin=0.5in]{geometry}
\usepackage{tabto}
\date{}
\author{hacker}
\begin{document}
	\title{Fundamentals of Computers}
	\maketitle
	
	\newpage
	
	\section{Defintion}
	A computer is an electronic device operating under the control of instructions, which tells the machine what to do. It is capable of excepting data (input) processing data arithmetically \& logically producing output from the processing and storing the results for the future use.
	
	\section{Characteristics of Computers}
	$\bullet$ Speed \\
	The computer process the data at an extreamly fast rate at millions or billions of instructions per second. The speed of the computer is calculated in MHz (Mega Hertz) i.e. 1 Million instructions per second.	
	\\ \\
	$\bullet$ Accuracy \\
	The computers are very accurate. The level of accuracy depends on the instructions and the types of machines we use. Since the computer is capable of doing only what it is instructed to do, faulty instructions for data processing may lead to faulty result. This also known as GIGO (Garbage In Garbage Out)
	\\ \\
	$\bullet$ Diligence \\
	Computer being a machine dosen't suffer from the human traits of tiredness, lack of concentration. If 4 million calculations have to be performed then the computer will perform the last 4 million calculations with same acuuracy and speed of the first one calculation.
	\\ \\
	$\bullet$ Storage Capability \\
	Computers can store large amounts of data and can recall the requierd information almost instantaneously. The main memouy of computer is relatively small and it can hold only certain amount of data. Therefore the data are stored on a secondary storage devices such as magnetic tapes or disks.
	\\ \\
	$\bullet$ Versitality \\
	Computers can quite versitile in nature, it can perform multiple tasks simulatneously with equal ease. \\
	Example : Compiling code and executing the precompiled code.
	\\ \\
	$\bullet$ Reliability
	It is the measurement of the performence of a computer which is measured against some pre-determined standard for operations without any failure. The major reason behind the eiliability of computer is that at hardware level. It does not requier any human intervention between it's processing operations. Computers have builtin diagnostic capabilities which help in the continues monitorinng of the system.
	\\ \\
	$\bullet$ Resource sharing \\
	In the intial stages if development computers used to be isolated machines with tremendous growth in computer technologies computers today have the capability of to connect with each other. Thsi has made the sharing of costly resources like printer possible.
	
	\section{Limitations of Computers}
	$ \bullet $ A computer can only perform what it is programmed to do. 
	\\ \\
	$ \bullet $ The computer needs well defined instructions to perform any operation, hence computers have unable to give any \\ \tabto{0.35cm} conclusion without going through intermidiate steps. 
	\\ \\
	$ \bullet $ A computers use is limited in areas where qualitatibe considerations are important, it can make plans based on \\ \tabto{0.35cm} situations and information but it cannot weather they will succeed.
	
	\newpage
	
	\section{Evolution of Computers}
	\subsection{Abacus}
	In the begining the tasks was simply counting or adding people used either fingers or pebbles along the lines in sand. \\
	In order to conviniently have the sand and pebbles all the time, people in Asia built a counting device called Abacus, this device allowed users to do calculations using system of sliding beans arranged on a rack. 
	\\ \\
	The device invented after the Abacus are Napier bones and Slide rule. 
	
	\subsection{Pascaline}
	In 1642, A French Mathematician Blaise Pascal invanted the first functional automatioc calculator, this is brass rectangular box called as Pascaline used 8 movable dials to add numbers upto 8 figures long.
	
	\subsection{Stepped Reckoner}
	In 1694, A German mathematician Godfried Wilhem Van Leibniz extended Pascal's design to perform multiplication, division and to find out square roots. \\
	It was the first man produced calculating device. which was designed to perform multiplication by repeated addition. \\
	Leibnitz mechanical multiplyer work by a system of gears and dials. The only problem with device was it lacked mechanical precision in it's constructions and was not reliable.
	
	\subsection{Difference Engine}
	In 1882, Charles Babbage English mathematician proffessor proposed a machine to perform differential equations. This machine is powered by steam engine and as large as locomotive. The machine would have a stored program and could perform calculations \& print the result automatically.
	
	\subsection{Analytical Engine}
	In 1833 Babbage concentrated on this machine, included input devices inthe form of perforated cards containing operating instructions and a store for memory or 1000 numbers upto 50 decimal digits long. It also contain a control unit to allow processing instructions in any sequence and output devices to produce printed results. Babbage borrowed the idea of punch cards to encode the machine insstruction from Joseph Marie Jacqards loom.
	
	\subsection{Hollerith's Tabulator}
	In 1899, Herman Hollerith who worked for the US census bureau who applied Jacquard loom concepts to computing. The method used cards to store the data which will be fed into to store the data which will be fed a machine that combined the results mechanically.
	
	\subsection{Mark I}
	Which was built as a partnership between Harvard, Aiken and IBM in 1994 this electronic calculating machines used relays \& electromagnetic componenet to replace mechanical components.
	
	\subsection{ENAIC}
	In 1946, John Eckert \& John Mauchly of Moore school of Engineering developed the electronic numerical integrator and calculator. This computer used electronic vaccum tubes to make the integral parts of the computer.
	
	\subsection{EDDAC}
	The next device proposed is electronic discrete variable automatic computer. It was the first electronic computer to use the stored program. concept introduced by John Boneuman, it also had capabilities of conditional transfer of control i.e. the computer could stop any time \& resume operations.
	
	\subsubsection{Full forms}
	\textbf{EDSAC} \\
	Electronic Delay Storage Automatic Calculator.
	\\ \\
	\textbf{UNIVAC} \\
	Universal Automatic Computer.
	\\ \\
	\textbf{LARC} \\
	Livermore Advanced Research Computer
	
	\subsection{Genartions of computers}
	\subsubsection{First Genaration}
	$\bullet$ These computers used vaccum tubes for circuitry and magnetic drums for memory.\\
	$\bullet$ A magnetic drum is a metal cylinder coated with magnetic FeO (Iron oxide) on which data \& programs can be stored. \\
	$\bullet$ The input was based on punch cards and paper tapes \& output was in the form of printers. \\
	$\bullet$ First genaration computers relied on binary coded languege. \\
	$\bullet$ Also called Machine level languege to perform operations. \\
	$\bullet$ These languege were able to solbe only one problem at a time each machine was fed with different binary \\
	\tabto{0.35cm} codes and hence were difficult to program this resulted in lack of versitality and speed.
	
	\subsubsection*{Charecteristics of First Genaration}
	$\bullet$ These computers were based on vaccum tubes technology. \\
	$\bullet$ These were fastest computing devices of their times with milisecond precision, these computers were very \\ \tabto{0.35cm}large and requiered a lot of space for installation \\
	$\bullet$ Since thousands of vaccum tubes were used they genarated a large amount of heat, hence air conditioning was essential \\
	$\bullet$ Tubes were non-portable and very slow equipments \\
	$\bullet$ They lacked in versitality and speed \\
	$\bullet$ They were expensive to operate \& used a large amount of electricity. \\
	$\bullet$ These machines were unreliable in and prone to frequent hardware failure, hence constant maintenence was necessary \\
	$\bullet$ Since machine languege was used used these computers were very difficult to program \& use \\
	$\bullet$ Each individual component has to be assembled manually, hence commercial appeal of these computers were poor
	
	\subsubsection{Second Genaration}
	$\bullet$ They used transistors which were superior to vaccum tubes. \\
	$\bullet$ A transistor is made up of semiconductor material like Germanium \& Silicon \\
	$\bullet$ It usually had 3 lids \& performs electrical functions such as voltage, power amplication with low power consumption \\ \tabto{0.35cm}or requierment \\
	$\bullet$ In these computers, magnetic drums were used as primary memory and magnetic disks as secondary storage devices \\
	$\bullet$ Input will be provided on punch cards and printout for the output. The major development of thid gen includes the \\ \tabto{0.35cm}program from machine languege from assembly languege. Assembly languege uses mnemonics for instruction rather \\ \tabto{0.35cm}than numbers as a result programming become less combursome \\
	$\bullet$ Early high level languege like COBOL \& Fortran also came into existance in this period.
	\\ \\
	\textbf{Example} \\
	DDP-8, IBM 1401. IBM 7090
	
	\subsubsection*{Charecteristics of Second Genaration}
	$\bullet$ These machines were based on transistor same technology. \\
	$\bullet$ The computational time of these computers was reduced to micro seconds from miliseconds, these were more reliable \\ \tabto{0.35cm} and less prone to hardware failure, hence they requiered less frequent maintenance. \\
	$\bullet$ The systems were more portable and and genarated less amount of heat. But computer still required less heat, \\ \tabto{0.35cm} But they still requiered air conditioning \\
	$\bullet$ Manula assembly of individual units into a fuctional units was still requiered.
	
	\subsection{Third Genaration}
	$\bullet$ The development of integrated circuit is also called as IC, it was the core technology used in this genaration. \\
	$\bullet$ An IC consist of a single chip with many components such as transistors and resistors fabricated on it. \\
	$\bullet$ IC's replaced several individually used wired transistors. \\
	$\bullet$ This made computers smaller in size and reliable \& efficient. \\
	$\bullet$ Insted of puch cards and printouts user interacted with the operating system. \\
	$\bullet$ This allow the device to run many different applications symultaneously with central program that monitor the memory.
	\\ \\
	\textbf{Example} \\
	NCR 395, IBM 360, IBM 370
\end{document} 
