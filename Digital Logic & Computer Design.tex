\documentclass{article}
\author{hacker}
\usepackage[a4paper, margin=0.5in]{geometry}
\begin{document}
	\title{DIGITAL LOGIC AND COMPUTER DESIGN}
	\maketitle
	
	$[$ \textbf{Note} $]$ : Originally written by M.Morris Morvo and modified by author
	
	\newpage
	
	\section{Number System}
	\subsection{Definition}
	A number system is writing system used to represent a number. \\
	$[$\textbf{Note}$]$ : A number is used to represent counts, quantities and to perform mathematical operations.
	\\ \\
	\textbf{There are four types of number systems :} \\
	$ \bullet $ Decimal number system \\
	$ \bullet $ Binary number system \\
	$ \bullet $ Octal number system \\
	$ \bullet $ Hexadecimal number system
	
	\subsection{Decimal number system}
	It is considered as base 10 number system. It supports 10 digits starting from 0 to 9 i.e A decimal number is formed by a sequence of these 10 digits.
	\\ \\
	\textbf{A decimal number system is based on 3 things :} \\
	$ \bullet $ The digit \\
	$ \bullet $ The position of digit in the number. \\
	$ \bullet $ The number base.
	
	\subsubsection{Example}
	1415.54 \\ \\
		\begin{tabular}{|c|c|c|c|c|c|c|c|}
			\hline
			\textbf{Weight}	& $ {10}^{3^{^{^{}}}} $ & $ 10^2 $ & $ 10^1 $ & $ 10^0 $ & & $ {10}^{-1} $ & $ {10}^{-2} $ \\ \hline
			\textbf{Position} & 3 & 2 & 1 & 0 & & $ -1 $ & $ -2 $
			\\ \hline
			\textbf{Number} & 1 & 4 & 1 & 5 & . & 5 & 4 
			\\ \hline
		\end{tabular}
	\\ \\ \\
	i.e. $ 1 \times 10^3 + 4 \times 10^2 + 1 \times 10^1 + 5 \times 10^0 + 5 \times 10^{-1} + 4 \times 10^{-2} = \underline{\underline{1415.54}}$ 
	
	\subsection{Binary Number System}
	It is considered as base 2 number system. It supports only two digits 0's and 1's.
	
	\subsubsection{Example}
	$ 1001.11_{(2)} $
	
	\subsection{Octal Number System}
	It is considered as base 8 number system. It supports only eight digits starting from 1 to 7.
	
	\subsubsection{Example}
	$ 143.56_{(8)} $
	
	\subsection{Hexadecimal Number System}
	It is considerd as base 16 number system. It supports 16 digits starting from 0-9 and A-F.
	\\
	where A=10, B=11 $ \dots $ F=15
	\subsubsection{Example}
	BCA$ .12_{(16)} $
	
	\newpage
	
	\section{Number System Conversion}
	\subsection{Converting Decimal to Binary}
	\subsubsection{Converting decimal integer to binary}
	\textbf{The procedure :} \\
	\textbf{Step 1:} Divide a given decimal integer by 2 and mark the remainder value.\\
	\textbf{Step 2:} Repeat Step 1 untill the quotient becomes zero.\\
	\textbf{Step 3:} Read the number from bottom to top i.e the equivalent binary value for the given decimal integer.
	\\ \\
	\textbf{Example} \\
	$ 145_{(10)} $ to $ ()_{(2)} $
	\\ \\
	$2|\underline{145 ~~~~} \\
	 2|\underline{72 - 1} \\
	 2|\underline{36 - 0} \\
	 2|\underline{18 - 0} \\
	 2|\underline{9 - 0} \\
	 2|\underline{4 - 1} \\
	 2|\underline{2 - 0} \\
	 2|\underline{1 - 0} \\
	 {~}|\underline{0 - 1} $
	 \\ \\
	 $ \Longrightarrow \underline{\underline{10010001_{(2)}}} $ 
	 \\ \\
	 $ 543_{(10)} $ to $ ()_{(2)} $
	 \\ \\
	 $2|\underline{543} \\
	 2|\underline{271 - 1} \\
	 2|\underline{135 - 1} \\
	 2|\underline{67 - 1} \\
	 2|\underline{33 - 1} \\
	 2|\underline{16 - 1} \\
	 2|\underline{8 - 0} \\
	 2|\underline{4 - 0} \\
	 2|\underline{2 - 0} \\
	 2|\underline{1 - 0} \\
	 {~}|\underline{0 - 1} 
	 $
	 \\ \\
	 $ \Longrightarrow \underline{\underline{1000011111_{(2)}}} $
	 \\ \\
	 $ 543_{(10)} $ to $ ()_{(2)} $
\end{document}