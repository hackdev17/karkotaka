% 
% 
% _________________________   ______  ______________________  _________
% __  ____/__  ____/__  __/   ___   |/  /__    |__  __/__  / / /_  ___/
% _  /    __  __/  __  /      __  /|_/ /__  /| |_  /  __  /_/ /_____ \ 
% / /___  _  /___  _  /       _  /  / / _  ___ |  /   _  __  / ____/ / 
% \____/  /_____/  /_/        /_/  /_/  /_/  |_/_/    /_/ /_/  /____/
% 
% --------------------------------------------------------------------
% Exam Details
% --------------------------------------------------------------------
% Exam date    :  16-06-2022
% Exam version :  D-4
% --------------------------------------------------------------------
% 
% 
% --------------------------------------------------------------------
% Document Details
% --------------------------------------------------------------------
% Author       :  Written dy hacker
% Copying      :  Allowed
% Licence      :  GPLv3
% --------------------------------------------------------------------
% 
% 
% --------------------------------------------------------------------
% Important Copyright Message
% --------------------------------------------------------------------
% This source is licensed under GNU GPL version 3 or later.
% This is free software: you are free to change and redistribute
% it.
% There is NO WARRANTY, to the extent permitted by law.
% Please visit https://gnu.org/licenses/gpl.html for more details
% --------------------------------------------------------------------
% 
% 
1. If A is a matrix of order 3 \times 3, then \left( A^2\right)^{-1} is equal to
(A) \left(A^{-1}\right)^2 \qquad \qquad \qquad \qquad \qquad (B) A^2
(C) \left(-A\right)^{-2} \qquad \qquad \qquad \qquad \qquad (D) \left(-A^2\right)^2

2. If A =   \left[ \begin{array}{cc} 2 & -1 \\ 3 & -2 \end{array}\right] , then inverse of the matrix A^3 is
(A) -1 \qquad \qquad \qquad \qquad \qquad (B) 1
(C) -A \qquad \qquad \qquad \qquad \qquad (D) A

3. If A is a skew symmetric matrix, then A^{2021} is
(A) \text{Column matrix} \qquad \qquad \qquad \qquad \qquad \qquad \qquad (B) \text{Symmetric matrix}
(C) \text{Skew symmetric matrix} \qquad \qquad \qquad \qquad \qquad (D) \text{Row matrix}

4. If A =   \left[ \begin{array}{cc} 0 & 1 \\ 0 & 0 \end{array}\right] then \left(aI+bA\right)^n is (where I is the identity matrix of order 2)
(A) a^nI+n.a^{n-1}b.A \qquad \qquad \qquad \quad (B)  a^nI+na^nbA
(C) a^nI+b^nA \qquad \qquad \qquad \qquad \qquad (D)  a^2I+a^{n-1}b.A

5. If A is a 3 \times 3 matrix such that \left\lvert 5. \text{adj} A \right\rvert = 5 then |A| is equal to
(A) \pm \frac{1}{25} \qquad \qquad \qquad \qquad \quad (B) \pm \frac{1}{5}
(C) \pm 5 \qquad \qquad \qquad \qquad \qquad (D) \pm 1

6. If there are two values of 'a' which makes determinant \Delta = \left\lvert \begin{array}{ccc} 1 & -2 & 5 \\  2 & a & -1 \\ 0 & 4 & 2a \end{array} \right\rvert = 86 , Then the sum of these numbers is
(A) 9 \qquad \qquad \qquad \qquad \qquad (B) 4
(C) 5 \qquad \qquad \qquad \qquad \qquad (D) -4

7. If the vertices of triangle are (-2,6) ~ (3,-6) and (1,5) , then the area of the triangle is
(A) 15.5 ~ \text{sq. units} \qquad \qquad \qquad \qquad \quad (B) 30 ~ \text{sq. units}
(C) 35 ~ \text{sq. units} \qquad \qquad \qquad \qquad \qquad (D) 40 ~ \text{sq. units}

8. Domain of \cos^{-1}[x] is, where [~] denotes a greatest integer function
(A) (-1,2) \qquad \qquad \qquad \qquad \qquad (B) [-1,2]
(C) [-1,2) \qquad \qquad \qquad \qquad \qquad (D) (-1,2]

9. If y=\left( 1+x\right)^2 \tan^{-1} x - x , then \frac{dy}{dx} is
(A) \frac{\tan^{-1} x}{x} \qquad \qquad \qquad \qquad \qquad (B) x^2 \tan^{-1} x
(C) x \tan^{-1} x \qquad \qquad \qquad \qquad \quad (D) 2x \tan^{-1} x

10. If x=e^{\theta} \sin\theta & y=e^{\theta} \cos\theta , Where \theta is a parametere, then \frac{dy}{dx} at (1,1) is equal to
(A) \frac{1}{2} \qquad \qquad \qquad \qquad \qquad (B) -\frac{1}{2}
(C) -\frac{1}{4} \qquad \qquad \qquad \qquad \quad (D) 0

11. If y=e^{\sqrt{x \sqrt{x \sqrt{x}}} \dots} & x>1 , then \frac{d^2y}{dx^2} at x= \log_{e} 3 is
(A) 5 \qquad \qquad \qquad \qquad \qquad (B) 0
(C) 1 \qquad \qquad \qquad \qquad \qquad (D) 3

12. If f(1)=1 & f'(1)=3 , then the derivative of f(f(f(x)))+(f(x))^2 at x=1 is
(A) 33 \qquad \qquad \qquad \qquad \qquad (B) 35
(C) 12 \qquad \qquad \qquad \qquad \qquad (D) 10

13. If y=x^{\sin x}+(\sin x)^x , then derivative \frac{dy}{dx} at x =\frac{\pi}{2} is
(A) \pi \log \frac{\pi}{2} \qquad \qquad \qquad \qquad \qquad (B) 1
(C) 1 \qquad \qquad \qquad \qquad \qquad \qquad ~ (D) \frac{4}{\pi}

14. If A_n = \left[ \begin{array}{cc} 1-n & n \\ n & 1-n \end{array}\right] , then |A_1| + |A_2| + |A_3| + \dots + |A_{2021}| =
(A) -(2021)^2 \qquad \qquad \qquad \qquad \qquad (B) (2021)^2
(C) 4042 \qquad \qquad \qquad \qquad \qquad \qquad ~ (D) -2021

15. The function f(x)= \log (1+x) - \frac{2x}{2+x} is increasing on
(A) (\infty , -1) \qquad \qquad \qquad \qquad \qquad (B) (-1, \infty )
(C) (- \infty , 0) \qquad \qquad \qquad \qquad \qquad (D) (-\infty , \infty)

16. The co-ordinates of the point on the \sqrt x+\sqrt y=6 at which tangent is equally inclined to the axis is
(A) (1,1) \qquad \qquad \qquad \qquad \qquad (B) (9,9)
(C) (6,6) \qquad \qquad \qquad \qquad \qquad (D) (4,4)

17. The function f(x)=4\sin^3 x - 6\sin^2 x+12\sin x +100 is strictly
(A) \text{decreasing in} \left[ 0,\frac{\pi}{2} \right] \qquad \qquad \qquad \qquad \qquad (B) \text{increasing in} \left( \pi,\frac{3\pi}{2} \right)
(C) \text{decreasing in} \left( \frac{\pi}{2},\pi \right) \qquad \qquad \qquad \qquad \quad ~ (D) \text{decreasing in} \left[-\frac{\pi}{2},\frac{\pi}{2} \right]

18. If [x] is the greatest integer function not greater than x then \int_{0}^{8} [x]~dx is equal to
(A) 30 \qquad \qquad \qquad \qquad \qquad (B) 29
(C) 20 \qquad \qquad \qquad \qquad \qquad (D) 28

19. \int_{0}^{\frac{\pi}{2}} \sqrt{\sin \theta} \cos^3\theta ~ d\theta is equal to
(A) \frac{7}{23} \qquad \qquad \qquad \qquad \qquad (B) \frac{8}{21}
(C) \frac{7}{21} \qquad \qquad \qquad \qquad \qquad (D) \frac{8}{23}

20. If e^y +xy =e the ordered pair \left( \frac{dy}{dx},\frac{d^2y}{dx^2} \right) at x=0 is equal to
(A) \left(\frac{-1}{e},\frac{-1}{e^2} \right) \qquad \qquad \qquad \qquad \qquad (B) \left(\frac{1}{e},\frac{-1}{e^2} \right)
(C) \left(\frac{-1}{e},\frac{1}{e^2} \right) \qquad \qquad \qquad \qquad \qquad ~ (D) \left(\frac{1}{e},\frac{1}{e^2} \right)

21. \int \frac{\cos 2x - \cos 2\alpha}{\cos x - \cos \alpha} ~ dx is equal to
(A) 2(\sin x+x \cos \alpha) + C \qquad \qquad \qquad \qquad \qquad (B) 2(\sin x-2x \cos \alpha) + C
(C) 2(\sin x+2x \cos \alpha) + C \qquad \qquad \qquad \qquad \quad ~ (D) 2(\sin x-x \cos \alpha) + C

22. If \int_{0}^{1} \frac{x e^x}{(2+x)}^3 ~ dx is equal to
(A) \frac{1}{27}.e+ \frac{1}{8} \qquad \qquad \qquad \qquad \qquad (B) \frac{1}{9}.e+ \frac{1}{4}
(C) \frac{1}{9}.e- \frac{1}{4} \qquad \qquad \qquad \qquad \qquad ~ (D) \frac{1}{27}.e- \frac{1}{8}

23. If \int \frac{dx}{(x+2)(x^2+1)} ~ dx = a \log |1+x^2|+b \tan^{-1} x+ \frac{1}{5} \log |x+2| + C , then
(A) a=\frac{1}{10} , b=\frac{2}{5} \qquad \qquad \qquad \qquad \qquad (B) a=\frac{-1}{10} , b=\frac{-2}{5}
(C) a=\frac{1}{10} , b=\frac{-2}{5} \qquad \qquad \qquad \qquad \qquad (D) a=\frac{-1}{10} , b=\frac{2}{5}

24. Area of the region bounded by the curve y=\tan x , the x - axis and the line x=\frac{\pi}{3} is
(A) \log 2 \qquad \qquad \qquad \qquad \qquad \quad (B) 0
(C) -\log 2 \qquad \qquad \qquad \qquad \qquad (D) \log \frac{1}{2}

25. Evaluate \int_{2}^{3} x^2 ~ dx as a limit of sum
(A) \frac{53}{9} \qquad \qquad \qquad \qquad \qquad (B) \frac{25}{7}
(C) \frac{19}{3} \qquad \qquad \qquad \qquad \qquad (D) \frac{72}{6}

26. \int_{0}^{\frac{\pi}{2}} \frac{ \cos x\sin x}{1+\sin x} ~ dx is equal to
(A) \log 2 \qquad \qquad \qquad \qquad \quad ~ (B) -\log 2
(C) 1-\log 2 \qquad \qquad \qquad \qquad (D) \log 2-1

27. If \frac{dy}{dx}=x^2 , then 2y(2) - y(1)=
(A) \frac{15}{4} \qquad \qquad \qquad \qquad \qquad (B) \frac{9}{4}
(C) \frac{13}{4} \qquad \qquad \qquad \qquad \qquad (D) \frac{11}{4}

28. The solution of differential equation \frac{dy}{dx}=(x+y)^2 is
(A) \tan^{-1} (x+y)=0 \qquad \qquad \qquad \qquad \qquad (B) \cot^{-1} (x+y)=C
(C) \cot^{-1} (x+y)=x+C \qquad \qquad \qquad \qquad (D) \tan^{-1} (x+y)=x+C

29. If y(x) be the solution of differential equation x \log x \frac{dy}{dx}+y=2x \log x , y(e) is equal to
(A) 0 \qquad \qquad \qquad \qquad \qquad (B) 2
(C) 2e \qquad \qquad \qquad \qquad \qquad (D) e

30. If |\vec{a}|=2 & |\vec{b}|=3 and the angle between \vec{a} & \vec{b} is 120 ~ \mathring{} , then the length of the vector \left| \frac{\vec{a}}{2} -\frac{\vec{b}}{3} \right| is
(A) 3 \qquad \qquad \qquad \qquad \qquad (B) \frac{1}{6}
(C) 1 \qquad \qquad \qquad \qquad \qquad (D) 2

31. If |\vec{a} \times \vec{b}|+|\vec{a} \cdot \vec{b}|^2=36 and |\vec{a}|=3 , then |\vec{b}| is equal to
(A) 36 \qquad \qquad \qquad \qquad \qquad (B) 4
(C) 2 \qquad \qquad \qquad \qquad \qquad ~ (D) 9

32. If \vec{\alpha}= \hat{i}-3\hat{j} & \vec{\beta}=\hat{i}+2\hat{j}-\hat{k} , then express \vec{\beta} in the form \vec{\beta}= \vec{\beta_1}+\vec{\beta_2}. Where \vec{\beta_1} is parallel to \vec{\beta_2} and \vec{\beta_2} is perpendicular to \vec{\alpha} then \vec{\beta_1} is given by
(A) \frac{5}{8}(\hat{i}+3\hat{j}) \qquad \qquad \qquad \qquad \qquad ~ (B) \hat{i}-3\hat{j}
(C) \hat{i}+3\hat{j} \qquad \qquad \qquad \qquad \qquad \quad ~ ~ (D) \frac{5}{8}(\hat{i}-3\hat{j})

33. The sum of the degree and the order of the differential equation (1+y_{1}^{2})^{\frac{3}{2}}=y_2
(A) 6 \qquad \qquad \qquad \qquad \qquad (B) 5
(C) 7 \qquad \qquad \qquad \qquad \qquad (D) 4

34. The co-ordinates of the foot of the perpendicular drawn from the origin to plane 2x-3y+4z=29 are
(A) (2,-3,-4) \qquad \qquad \qquad \qquad \qquad (B) (2,-3,4)
(C) (-2,-3,4) \qquad \qquad \qquad \qquad \qquad (D)(2,3,4)

35. The angle between pair of lines \frac{x+3}{3}=\frac{y-1}{5}=\frac{z+3}{4} and \frac{x+1}{1}=\frac{y-4}{4}=\frac{z-5}{2} is
(A) \theta=\cos^{-1} \left[\frac{8\sqrt{3}}{15} \right] \qquad \qquad \qquad \qquad \qquad (B) \theta=\cos^{-1} \left[\frac{19}{21} \right]
(C) \theta=\cos^{-1} \left[\frac{5\sqrt{3}}{16} \right] \qquad \qquad \qquad \qquad \qquad (D) \theta=\cos^{-1} \left[\frac{27}{5} \right]

36. The corner points of feasible region of an LPP are (0,2) , (3,0) , (6,0) , (6,8) & (0,5) then the minimum value of Z=4x+6y occurs at
(A) \text{infinite numbers of points} \qquad \qquad \qquad (B) \text{ only one point}
(C) \text{only two points} \qquad \qquad \qquad \qquad \qquad ~ (D) \text{finite numbers of points}

37. A dietician has to develop a special diet using two foods X & Y. Each packet (containing 30g) of food. X contains 12 units of calcium, 4 units of iron, 6 units of cholestrol and 6 units of vitamin A. Each packet of same quantity of food Y contains 3 units of calcium, 20 units of iron, 4 units of cholestrol and 3 units of vitamin A. The diet requiers atleast 240 units of calcium, 460 units of iron and and almost 300 units of cholestrol. The corner points of the feasible region are
(A) (2,72) , (15,20) , (0,23) \qquad \qquad \qquad \qquad \qquad ~ (B) (0,23) , (40,15) , (2,72)
(C) (2,72) , (40,15) , (115,0) \qquad \qquad \qquad \qquad \qquad (D) (2,72) , (40,15) , (15,20)

38. The distance of the point whose position vector is (2\hat{i}+\hat{j}-\hat{k}) from the plane \vec{r}\cdot(\hat{i}-2\hat{j}-a\hat{k})=4 is
(A) 8\sqrt{21} \qquad \qquad \qquad \qquad \qquad (B) \frac{-8}{\sqrt{21}}
(C) \frac{-8}{21} \qquad \qquad \qquad \qquad \qquad \quad (D) \frac{8}{21}

39. Find the mean number of heads in three tosses of a fair coin :
(A) 4.5 \qquad \qquad \qquad \qquad \qquad (B) 2.5
(C) 3.5 \qquad \qquad \qquad \qquad \qquad (D) 1.5

40. If A and B are two events such that P(A)=\frac{1}{2} , P(B)=\frac{1}{3} & P(A|B)=\frac{1}{4} , then P(A^{\prime } \cap B^{\prime}) is
(A) \frac{3}{16} \qquad \qquad \qquad \qquad \qquad (B) \frac{1}{12}
(C) \frac{3}{4} \qquad \qquad \qquad \qquad \qquad ~ (D) \frac{1}{4}

41. A pandemic has been spreading all over the world. The probabilities are 0.7 and there will be lockdown, 0.8 that of pandemic is controlled in one month if there is a lockdown and 0.3 that of it is controlled in one month if there is no lockdown. The probability that of pandemic controlled in one month is
(A) 1.65 \qquad \qquad \qquad \qquad \qquad (B) 1.46
(C) 0.46 \qquad \qquad \qquad \qquad \qquad (D) 0.65

42. If A and B are two independent events such that P(\overline{A})=0.75 , P(A \cup B)=0.65 & P(B)=x , then find x
(A) \farc{8}{15} \qquad \qquad \qquad \qquad \qquad (B) \farc{9}{14}
(C) \farc{7}{15} \qquad \qquad \qquad \qquad \qquad (D) \farc{5}{14}

43. Suppose that the number of elemets in a set A is p, the number of elemet inn the set B is q and the number of elemets n the set A \times B is 7 , then p^2+q^2 = \underline{\qquad \qquad}
(A) 51 \qquad \qquad \qquad \qquad \qquad (B) 42
(C) 49 \qquad \qquad \qquad \qquad \qquad (D) 50

44. The domain of the function f(x)=\frac{1}{\log_{10}(1-x)}+\sqrt{x+2} is
(A) [-2,1) \qquad \qquad \qquad \qquad \qquad (B) [-2,0)
(C) [-2,1) \cup (0,1) \qquad \qquad \qquad ~~ (D) [-2,1) \cap (0,1)

45. The trignometric function y=\tan x in the II quadrant
(A) \text{decreases from} ~ -\infty ~ \text{to} ~ 0 \qquad \qquad \qquad \qquad \qquad (B) \text{increases from} ~ \infty ~ \text{to} ~ 0
(C) \text{increases from} ~ -\infty ~ \text{to} ~ 0 \qquad \qquad \qquad \qquad \qquad (D) \text{decreases from} ~ \infty ~ \text{to} ~ 0

46. The degree measure of \frac{\pi}{32} is equal to
(A) 5\mathring{}37~^{\prime}~20^{\prime \prime} \qquad \qquad \qquad \qquad \qquad (B) 5\mathring{}~37^{\prime}~30^{\prime \prime}
(C) 4\mathring{}~30^{\prime}~30^{\prime \prime} \qquad \qquad \qquad \qquad \qquad (D) 5\mathring{}~30^{\prime}~20^{\prime \prime}

47. The value of \sin \frac{5\pi}{12}\sin \frac{\pi}{12} is
(A) 1 \qquad \qquad \qquad \qquad \qquad (B) \farc{1}{2}
(C) \frac{1}{4} \qquad \qquad \qquad \qquad \qquad (D) 0

48. \sqrt{2+\sqrt{2+\sqrt{2+2\cos 8\theta}}}=
(A) 2\cos \theta \qquad \qquad \qquad \qquad \qquad (B) 2\sin \theta
(C) 2\cos \frac{\theta}{2} \qquad \qquad \qquad \qquad \qquad (D) \sin 2\theta

49. If A={1,2,3, \dots 10} then the number of subsets of A containing only odd numbers is
(A) 27 \qquad \qquad \qquad \qquad \qquad (B) 32
(C) 30 \qquad \qquad \qquad \qquad \qquad (D) 31

50. If all permutations of letters of the word MASK are arranged in the order as in dictionay with or without meaning, which one of the following is 19^{\text{th}} word ?
(A) \text{SAMK} \qquad \qquad \qquad \qquad \qquad (B) \text{AKMS}
(C) \text{AMSK} \qquad \qquad \qquad \qquad \qquad (D) \text{KAMS}

51. If a_1,a_2, \dots a_{10} is a geometric progression and \frac{a_3}{a_1}=25 , then \frac{a_9}{a_5} equals
(A) 5^4 \qquad \qquad \qquad \qquad \qquad \qquad (B) 5^3
(C) 2(5^2) \qquad \qquad \qquad \qquad \qquad ~~ (D) 3(5^2)

52. If the staraight line 2x-3y+17=0 is perpendicular to the line passing through the points (7,17) and (15,\beta) , then \beta equals
(A) 5 \qquad \qquad \qquad \qquad \qquad ~~ (B) 29
(C) -29 \qquad \qquad \qquad \qquad \qquad (D) -5

53. The octant in which the point (2,-4,-7) lies is
(A) \text{Third} \qquad \qquad \qquad \qquad \qquad (B) \text{Fourth}
(C) \text{Fifth} \qquad \qquad \qquad \qquad \qquad (D) \text{Eighth}

54. If f(x)= \left\{ \begin{array}{c} x^2-1 , \qquad 0<x<2 \\ 2x+3 , \qquad 2\le x<3 \end{array} , \right the quadratic equation whose roots are \lim_{x \rightarrow 2^{-}} f(x) and \lim_{x \rightarrow 2^{+}} f(x) is
(A) x^2-10x+21=0 \qquad \qquad \qquad \qquad \qquad (B) x^2-6x+9=0
(C) x^2-7x+8=0 \qquad \qquad \qquad \qquad \qquad ~~ (D) x^2-14x+49=0

55. If 3x+\iota(4x-y)=6-\iota where x and y are real numbers, then the values of x and y are
(A) 2,4 \qquad \qquad \qquad \qquad \qquad (B) 2,9
(C) 3,4 \qquad \qquad \qquad \qquad \qquad (D) 3,9

56. If the standard deviation of the numbers -1,0,1,k is \sqrt{5} where k>0 , then k is equal to
(A) \sqrt{6} \qquad \qquad \qquad \qquad \qquad ~ (B) 2\sqrt{\frac{10}{3}}
(C) 2\sqrt{6} \qquad \qquad \qquad \qquad \qquad (D) 4\sqrt{\frac{5}{3}}

57. If the set of x contains 7 elemets and set of y contains 8 elements, then the number of bijection from x to y is
(A) 8\text{P}_7 \qquad \qquad \qquad \qquad \qquad (B) 7!
(C) 8! \qquad \qquad \qquad \qquad \qquad \quad (D) 0

58. If f:\mathbb{R} \longrightarrow \mathbb{R} be defined by f(x)=\left\{ \begin{array}{c} 2x ~~ : ~~ x>3 \\ \qquad x^2 ~~ : ~~ 1<x\le3 \\ 3x ~~ : ~~ x\le1 \end{array} then f(-1)+f(2)+f(4) is
(A) 10 \qquad \qquad \qquad \qquad \qquad (B) 9
(C) 14 \qquad \qquad \qquad \qquad \qquad (D) 5

59. Let the relation R is defined in \mathbb{N} by a~R~b , if 3a+2b=27 then R is
(A) \left\{ \left( 0, \frac{27}{2}\right) , (1,12),(3,9),(5,6),(7,3) \right\} \qquad (B) \{(1,12),(3,9),(5,6),(7,3),(9,0)\}
(C) {(2,1),(9,3),(6,5),(3,7)} \qquad \qquad \qquad \qquad ~ (D) \{(1,12),(3,9),(5,6),(7,3)\}

60. \lim_{y \longrightarrow 0} \frac{\sqrt{3+y^3}-\sqrt{3}}{y^3}=
(A) \frac{1}{3\sqrt{2}} \qquad \qquad \qquad \qquad \qquad (B) 2\sqrt{3}
(C) 3\sqrt{2} \qquad \qquad \qquad \qquad \qquad (D) \frac{1}{2\sqrt{3}}

% --------------------------------The End--------------------------------
