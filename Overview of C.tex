\documentclass{article}
\usepackage[a4paper, margin=0.5in]{geometry}
\usepackage{fontspec}
\newfontfamily{\code}{Source Code Pro}
\usepackage{tabto}
\usepackage{amsmath}
\date{}
\author{hacker}

\begin{document}
	\title{Overview of C}
	\maketitle

	\newpage
	
	\section{History of C}
	\begin{center}
		(1960) ALGOL [International Group] \\
		$ \downarrow $ \\
		(1967) BCPL [Martin Richards] \\ 
		$ \downarrow $ \\
		(1970) B [Ken Thompson] \\ 
		$ \downarrow $ \\
		(1972) Traditonal C [Denis Ritchie]\\ 
		$ \downarrow $ \\
		(1978) K \& RC [Kernighan \& Ritchie] \\ 
		$ \downarrow $ \\
		(1989) ANSI C [ANSI commitee] \\ 
		$ \downarrow $ \\
		(1990) ANSI/ISO C [ISO commitee] \\
		$ \downarrow $ \\
		(1999) C99 [Standardization commitee]
	\end{center}

	\subsection{Full form}
	\textbf{BPCL} \\
	Basic Combined Peogramming Languege. \\
	\\
	\textbf{ANSI} \\
	American National Standard Institute \\
	\\
	\textbf{ISO} \\
	International Standard Organization
	
	\section{Importance of C}
	$ \bullet $ It is rhobust languege which has builtin functions and operators. \\
	$ \bullet $ It is a high level languege \\
	$ \bullet $ Program written in C are efficient and fast \\
	$ \bullet $  There are only 32 keywords in C and its strength lies in the built in functions (Library functions) \\
	$ \bullet $ C is a portable languege \\
	$ \bullet $ C is a supports structured programming \\
	$ \bullet $ C has ability to extended itself \\

	\section{Basic structure of C}
	$ \bullet $ Documentation section \\
	$ \bullet $ Link section \\
	$ \bullet $ Definition section \\
	$ \bullet $ Global declaration section \\
	$ \bullet $ main() function section \\
	{
		\tabto{0.5cm} \code main() \{ \\ 
			\tabto{1cm} declaration\_part \\ 
			\tabto{1cm} execution\_part \\ 
		\tabto{0.5cm} \} 
	} \\
	$ \bullet $ Subprogram section
	{
		\tabto{1cm} \code Function1 \\
		\tabto{1cm}Fuction2 \\
		\tabto{1.5cm} $ \vdots $ \\
		\tabto{1cm} Fuction\textit{n} \\
	}

	\newpage

	\subsection{Documentation section}
	It consist of a set of comment lines. Which will have the name of the program the author and other details.
	
	\subsection{Link section}
	It provides instructions to the compiler to line functions from the system library.
	
	\subsection{Definition section}
	It defines symbolic constants.
	
	\subsection{Global declaration section}
	There are some variables are called GDS and are declared in the Global declaration, i.e. GDS
	
	\subsection{main() section}
	Every C program must have one {\code main()} function section. This containes two parts,
	declaration part \& executable part. The declaration part containes all the variables. There is atleast one statement in executable part. All statements in declaration \& execution part end with semicolon (;)
	
	\subsection{Subprogram section}
	It containes all the user-defined functions that are called in the {\code main()} funcions.
	
	\subsubsection{Example}
	{
		\code \tabto{0.5cm} \#include<stdio.h>
		\tabto{0.5cm} main () \{
			\tabto{1cm} printf("Hello, Welcome to C !");
		\tabto{0.5cm} \}
	}

	{
		\tabto{0.5cm} \code \#include<stdio.h>
		\tabto{0.5cm} main () \{
			\tabto{1cm} int a,b,add;
			\tabto{1cm} printf("\textbackslash nEnter two numbers :\textbackslash n> ");
			\tabto{1cm} scanf("\%d\%d", \&a, \&b);
			\tabto{1cm} add=a+b;
			\tabto{1cm} printf("\textbackslash n The sum of two numbers is : \%d", add);
		\tabto{0.5cm} \}
	}

	\subsection{Executing a C program}
	$ \bullet $ Creating a program \\
	$ \bullet $ Compiling a program \\
	$ \bullet $ Linking a program with the function \\
	$ \bullet $ Executing a program
\end{document}
